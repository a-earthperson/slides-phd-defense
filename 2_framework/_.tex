\section{Research Objective 2: Develop data-parallel methods for evaluating Boolean circuits}
\begin{frame}
    \Large{\centerline{\textbf{Research Objective 2}}}
    \vspace{6pt}
    \large{\centerline{\textbf{Develop data-parallel methods for evaluating Boolean circuits}}}
\end{frame}

%% kernel and bitpack data layout

%% the sycl execution model

%% mapping data / dag to sycl kernels

% special case for k/n kernels
\begin{frame}[t]{Handling Voter Logic - Expansion} 
  \begin{columns}
    \column{0.60\textwidth}
    {
      \textbf{The Simplest Type of Query: Eval(G)}\par
      \begin{itemize}
          \item {Set the inputs [on/off].}
          \item {Observe the outputs [on/off].}
      \end{itemize}\par
      \vspace{2pt}
      \tiny{\textbf{\emph{{Can be used as a building block for an embedding ML model.}}}}\par
            \vspace{10pt}
      \normalsize\textbf{But how to handle K/N gates?}

    }
     \column{0.4\textwidth}
      \begin{figure}[h!]
    \tikzset{
    grow'=down,
    level distance=42pt,
    % sibling distance=24pt,
    % sibling distance=8mm/#1,
    level/.style={sibling distance=3pt+44pt/(#1*#1*0.75)},
    edge from parent/.append style={
        draw,
        %thick,
        edge from parent path={
        (\tikzparentnode.west) |- ($(\tikzparentnode.west)!0.5!(\tikzchildnode.east)$) -| (\tikzchildnode.east)
        },
    },
    every node/.append style={
        anchor=center,
        rotate=90,
        draw=black,
        fill=white,
        thick,
        font=\footnotesize\bfseries,
        text centered,
        inner sep=0pt
    },
    var/.style={
        shape=circle,
        minimum height=16pt,
    },
    and/.style={
    and gate US,
    },
  not/.style={
    not gate US,
  },  
  or/.style={
    or gate US,
  },
  vot/.style={
    or gate US,
  }
}
    \centering
    \resizebox{\textwidth}{!}{%
\begin{tikzpicture}[
    circuit logic US,
    tiny circuit symbols,
    level 1/.style={sibling distance=5*18pt},
    level 2/.style={sibling distance=18pt},
    ]
  \node[or]{\rotatebox{-90}{Y}}
  % Subsets of size 3
  child { node[and] {} 
    child { node[var] {\rotatebox{-90}{$X_1$}} }
    child { node[var] {\rotatebox{-90}{$X_2$}} }
    child { node[var] {\rotatebox{-90}{$X_3$}} }
  }
  child { node[and] {} 
    child { node[var] {\rotatebox{-90}{$X_1$}} }
    child { node[var] {\rotatebox{-90}{$X_2$}} }
    child { node[var] {\rotatebox{-90}{$X_4$}} }
  }
  child { node[and] {} 
    child { node[var] {\rotatebox{-90}{$X_1$}} }
    child { node[var] {\rotatebox{-90}{$X_2$}} }
    child { node[var] {\rotatebox{-90}{$X_5$}} }
  }
  child { node[and] {} 
    child { node[var] {\rotatebox{-90}{$X_1$}} }
    child { node[var] {\rotatebox{-90}{$X_3$}} }
    child { node[var] {\rotatebox{-90}{$X_4$}} }
  }
  child { node[and] {} 
    child { node[var] {\rotatebox{-90}{$X_1$}} }
    child { node[var] {\rotatebox{-90}{$X_3$}} }
    child { node[var] {\rotatebox{-90}{$X_5$}} }
  }
  child { node[and] {} 
    child { node[var] {\rotatebox{-90}{$X_1$}} }
    child { node[var] {\rotatebox{-90}{$X_4$}} }
    child { node[var] {\rotatebox{-90}{$X_5$}} }
  }
  child { node[and] {} 
    child { node[var] {\rotatebox{-90}{$X_2$}} }
    child { node[var] {\rotatebox{-90}{$X_3$}} }
    child { node[var] {\rotatebox{-90}{$X_4$}} }
  }
  child { node[and] {} 
    child { node[var] {\rotatebox{-90}{$X_2$}} }
    child { node[var] {\rotatebox{-90}{$X_3$}} }
    child { node[var] {\rotatebox{-90}{$X_5$}} }
  }
  child { node[and] {} 
    child { node[var] {\rotatebox{-90}{$X_2$}} }
    child { node[var] {\rotatebox{-90}{$X_4$}} }
    child { node[var] {\rotatebox{-90}{$X_5$}} }
  }
  child { node[and] {} 
    child { node[var] {\rotatebox{-90}{$X_3$}} }
    child { node[var] {\rotatebox{-90}{$X_4$}} }
    child { node[var] {\rotatebox{-90}{$X_5$}} }
  }
  % Subsets of size 4
  child { node[and] {} 
    child { node[var] {\rotatebox{-90}{$X_1$}} }
    child { node[var] {\rotatebox{-90}{$X_2$}} }
    child { node[var] {\rotatebox{-90}{$X_3$}} }
    child { node[var] {\rotatebox{-90}{$X_4$}} }
  }
  child { node[and] {} 
    child { node[var] {\rotatebox{-90}{$X_1$}} }
    child { node[var] {\rotatebox{-90}{$X_2$}} }
    child { node[var] {\rotatebox{-90}{$X_3$}} }
    child { node[var] {\rotatebox{-90}{$X_5$}} }
  }
  child { node[and] {} 
    child { node[var] {\rotatebox{-90}{$X_1$}} }
    child { node[var] {\rotatebox{-90}{$X_2$}} }
    child { node[var] {\rotatebox{-90}{$X_4$}} }
    child { node[var] {\rotatebox{-90}{$X_5$}} }
  }
  child { node[and] {} 
    child { node[var] {\rotatebox{-90}{$X_1$}} }
    child { node[var] {\rotatebox{-90}{$X_3$}} }
    child { node[var] {\rotatebox{-90}{$X_4$}} }
    child { node[var] {\rotatebox{-90}{$X_5$}} }
  }
  child { node[and] {} 
    child { node[var] {\rotatebox{-90}{$X_2$}} }
    child { node[var] {\rotatebox{-90}{$X_3$}} }
    child { node[var] {\rotatebox{-90}{$X_4$}} }
    child { node[var] {\rotatebox{-90}{$X_5$}} }
  }
  % Subset of size 5
  child { node[and] {} 
    child { node[var] {\rotatebox{-90}{$X_1$}} }
    child { node[var] {\rotatebox{-90}{$X_2$}} }
    child { node[var] {\rotatebox{-90}{$X_3$}} }
    child { node[var] {\rotatebox{-90}{$X_4$}} }
    child { node[var] {\rotatebox{-90}{$X_5$}} }
  };
\end{tikzpicture}
    }
    \label{fig:example-3of5-voter-tree}
\end{figure}
\par % Replace with your image file
  \end{columns}
\end{frame}

%% throughput graph
\begin{frame}{Eval Query}
  \begin{columns}
    \column{0.60\textwidth}
    {
      \textbf{Eval Query Performance on GPUs:}
      \begin{itemize}
          \item {Latency: 200-300 $ms$ per graph pass.}
          \item {Throughput: VRAM bound (see plot).}
          \item {Benchmarked on Nvidia GTX 1660 [6GB].}
          \item {Graph sizes: from $\approx 50$ to $\approx 2000$ nodes.}
          \item {Evals: from 16M to 1B per node per pass.}
      \end{itemize}
      \vspace{10pt}
      \textbf{Q: Are these enough samples to estimate the Expectation Query?}
    }
     \column{0.4\textwidth}
        \centering
        \includesvg[height=0.9\textheight]{1_concepts/mem_allocation_lines_zoom.svg}
  \end{columns}
\end{frame}





\section{Research Objective 3: Develop data-parallel Monte Carlo methods for estimating probabilities}
\begin{frame}
    \Large{\centerline{\textbf{Research Objective 3}}}
    \vspace{6pt}
    \large{\centerline{\textbf{Develop data-parallel Monte Carlo methods for estimating probabilities}}}
\end{frame}

\subsection{Back to Working Example: One Initiating Event, Three Fault Trees, Six Basic Events, Five End States}
\begin{frame}
  \begin{columns}
    \column{0.66\textwidth}
      \includesvg[height=\textheight]{1_concepts/dag_pass_3.svg} % Replace with your image file
     \column{0.33\textwidth}
      \includesvg[width=\linewidth]{1_concepts/pra-model.svg} % Replace with your image file
  \end{columns}
\end{frame}


\begin{frame}[t, allowframebreaks]
\frametitle{Monte Carlo Sampling}
\begin{itemize}
  \item Rather than summing or bounding all combinations of failures, \emph{simulate} random draws of \(\mathbf{X}\).
  \item Each Monte Carlo iteration:
    \begin{enumerate}
      \item Sample \(x_1, x_2,\dots,x_n \overset{\text{i.i.d.}}{\sim} \prod p(x_i)\).
      \item Evaluate the Boolean function \(F(\mathbf{x})\) (cost is just logical gate evaluation).
      \item Collect whether \(F(\mathbf{x})=1\) (failure) or 0 (success).
    \end{enumerate}
  \item Repeating for many samples \(\{\mathbf{x}^{(1)}, \dots, \mathbf{x}^{(N)}\}\) yields a \emph{sample average} estimate of the probability.
  \item Benefits:
    \begin{itemize}
      \item Bypasses explicit inclusion-exclusion expansions.
      \item Straightforward to parallelize (evaluate each draw in separate threads or blocks).
    \end{itemize}
\end{itemize}
\end{frame}

\begin{frame}[t, allowframebreaks]
\frametitle{Estimator for the Expected Value (i.e., Probability)}
\begin{itemize}
  \item A Boolean function \(F(\mathbf{x})\) can be viewed as an indicator function: \(F(\mathbf{x}) \in \{0,1\}\).
  \item The event \(\{F(\mathbf{X})=1\}\) has probability \(\mathbb{E}[F(\mathbf{X})]\).
  \item \textbf{Monte Carlo estimator:}
    \[
      \widehat{P}_N
      \;=\;
      \frac{1}{N}\sum_{i=1}^N 
      F\!\bigl(\mathbf{x}^{(i)}\bigr),
    \]
    where each \(\mathbf{x}^{(i)}\) is a random draw from the input distribution.
  \item By the Law of Large Numbers,
    \[
      \lim_{N \to \infty}\;\widehat{P}_N
      \;=\;
      \Pr\bigl[F(\mathbf{X})=1\bigr],
      \quad \text{almost surely}.
    \]
  \item Error decreases at rate \(\mathcal{O}(1/\sqrt{N})\), analyzed via the Central Limit Theorem.
\end{itemize}
\end{frame}

\subsection{Building a Monte Carlo Estimator for Event Probabilities}
\begin{frame}[t, allowframebreaks]
\frametitle{Boolean Functions: Basic Concepts}
\begin{itemize}
  \item Let \(\mathbf{x} = (x_1, x_2, \dots, x_n)\) be a vector of \(n\) Boolean variables, each \(x_i \in \{0,1\}\).
  \item A \emph{Boolean function} is any map \(F(\mathbf{x}): \{0,1\}^n \to \{0,1\}\).
  \item Example: If \(F\) encodes “system fails,” then \(F(\mathbf{x}) = 1\) signifies a failure mode, where \(\mathbf{x}\) captures component states.
  \item Modeling perspective:
    \begin{itemize}
      \item AND, OR, NOT, \(k\)-of-\(n\) gates allow composing complex logic.  
      \item Each \(F\) can be evaluated deterministically if we know \(\mathbf{x}\).
    \end{itemize}
\end{itemize}
\end{frame}

\begin{frame}[t, allowframebreaks]
\frametitle{Exact Probability Estimation: Inclusion-Exclusion}
\begin{itemize}
  \item Suppose each \(x_i\) has a probability \(p_i = \Pr[x_i=1]\), assuming independence.
  \item We want \(\Pr[F(\mathbf{x}) = 1]\), which is
  \[
    \Pr\bigl[F(\mathbf{X})=1\bigr]
    \;=\; 
    \sum_{\mathbf{x}\in \{0,1\}^n}
      F(\mathbf{x}) 
      \prod_{i=1}^n
      \bigl[p_i^{\,x_i}(1-p_i)^{\,1-x_i}\bigr].
  \]
  \item For sets of events, using the \emph{inclusion-exclusion principle}:
  \[
    \Pr\Bigl(\bigcup_{i=1}^n E_i\Bigr)
    \;=\;
    \sum_{k=1}^n \;(-1)^{k+1} 
    \;\;\sum_{1\le i_1< \cdots < i_k\le n}
    \!\Pr\bigl(E_{i_1}\cap\dots\cap E_{i_k}\bigr).
  \]

\end{itemize}
\end{frame}

\begin{frame}[allowframebreaks]
\frametitle{Approximation Methods: REA and MCUB}
 For large \(n\), exact enumeration of subsets is exponential, making it impractical for large Boolean circuits.
 \vspace{8pt}
\begin{itemize}
  \item \textbf{Rare-Event Approximation (REA):}
    \begin{itemize}
      \item Assumes each event has small probability \(p_i \ll 1\).
      \item Overlaps (intersections of multiple failures) are deemed negligible.
      \item Probability of the union \(\approx \sum_{i} \Pr[E_i]\), ignoring higher-order terms.
    \end{itemize}
\framebreak
  \item \textbf{Min-Cut Upper Bound (MCUB):}
    \begin{equation}
    \label{eq:mcub_slides}
      \Pr\Bigl[\bigcup_{C \in \{\mathrm{MCS}\}} C\Bigr]
      \;\le\;
      \sum_{C \in \{\mathrm{MCS}\}}
      \;\prod_{b \in C} p_b ,
    \end{equation}
    \begin{itemize}
      \item Interprets each minimal cut set (MCS) as a distinct mechanism for failure.
      \item Sums (over)estimate total failure if MCSs share components.
      \item Often used as a conservative bound in safety analyses.
    \end{itemize}
  \vspace{6pt}
  \item Both methods reduce complexity but can misestimate the true probability when events are not truly rare or heavily intersect.
\end{itemize}
\end{frame}



\begin{frame}[t]
\frametitle{Boolean Derivatives: Definition and Interpretation}
\begin{itemize}
\item \textbf{Boolean Derivative Concept:}  
  For a Boolean function \(F(\mathbf{x})\) with \(\mathbf{x}=(x_1,\ldots,x_n)\), the derivative with respect to \(x_i\) is defined via XOR:
  \[
    \frac{\partial F}{\partial x_i}
    \;=\;
    F(x_i=0,\mathbf{x}_{-i})
    \;\oplus\;
    F(x_i=1,\mathbf{x}_{-i}),
  \]
  where \(\oplus\) denotes the exclusive-OR operation, and \(\mathbf{x}_{-i}\) are all variables except \(x_i\).
\item \textbf{Interpretation:}  
  \begin{itemize}
    \item \(\frac{\partial F}{\partial x_i}(\mathbf{x})=1\) whenever \emph{flipping} \(x_i\) changes the value of \(F\) under the specific configuration \(\mathbf{x}_{-i}\).  
    \item Captures \emph{sensitivity}: if \(\frac{\partial F}{\partial x_i}\) rarely equals \(1\), then \(F\) is robust to changes in \(x_i\).  
  \end{itemize}
\end{itemize}
\end{frame}

\begin{frame}[allowframebreaks]
\frametitle{Extension to Monte Carlo Estimation of Boolean Derivatives}
\begin{itemize}
  \item \textbf{Key Idea:} Estimate \(\mathbb{E}[\,\partial F / \partial x_i\,]\) by sampling random configurations \(\mathbf{x}^{(s)}\) of the Boolean inputs, then checking how \(F\) changes when \(x_i\) is flipped.
  \vspace{8pt}
  \item \textbf{Sampling Procedure:}
    \begin{enumerate}
      \item Draw \(\mathbf{x}^{(s)} = \bigl(x_1^{(s)},\dots,x_n^{(s)}\bigr)\) from the distribution of interest.  
      \item Form \(\mathbf{x}^{(s)} \oplus \mathbf{e}_i\) by flipping the \(i\)th coordinate.
      \item Compute:
        \[
          \frac{\partial F}{\partial x_i}\bigl(\mathbf{x}^{(s)}\bigr)
          \;=\;
          F\!\bigl(\mathbf{x}^{(s)}\bigr)
          \;\oplus\;
          F\!\bigl(\mathbf{x}^{(s)} \oplus \mathbf{e}_i\bigr).
        \]
    \end{enumerate}
  \item \textbf{Insight:}  
    \begin{itemize}
    \item{Sensitivity and importance analysis using sampling methods.}
    \item{Gradient computation opens a path towards learning-based tasks.}
    \end{itemize}
\end{itemize}
\end{frame}

\begin{frame}[t, allowframebreaks]
\frametitle{Avoiding Inclusion-Exclusion via Monte Carlo}
\begin{itemize}
  \item Exact expansions for large circuits require enumerating all subsets of failing components or gates, which is computationally huge.
  \item In contrast, \emph{Monte Carlo} draws a sample \(\mathbf{x}\in \{0,1\}^n\) and directly evaluates \(F(\mathbf{x})\) without enumerating \emph{all} subsets.
  \item Each run picks a single draw of failed components from the distribution. After many runs, the frequency of \(F=1\) approximates its probability.
  \item Results:
    \begin{itemize}
      \item No exponential blow-up in the number of terms.
      \item Straightforward extension to complex gate structures, correlated variables.
      \item Parallelizable on modern CPU/GPU architectures.
    \end{itemize}
\end{itemize}
\end{frame}

\begin{frame}[t, allowframebreaks]
\frametitle{Data-Parallel Implementation using SYCL}
\item \textbf{Data-Parallel Monte Carlo for Boolean Circuits:}
  \begin{itemize}
    \item{Simultaneous evaluation of \emph{all} intermediate gates, success, and failure paths.}
    \item{Relax coherence constraints - arbitrary shapes with NOT gates permitted.}
    \item{Vectorized bitwise hardware ops for logical primitives (AND, OR, XOR, etc.)}
    \item{Specialized treatment of \(k/n\) logic, without expansion.}
    \item{Simultaneous use of all available compute - GPUs, multicore CPUs.}
    \vspace{16pt}
  \end{itemize}
\end{frame}