\section{Research Objective 3: Develop data-parallel Monte Carlo methods for estimating probabilities}
\begin{frame}
    \Large{\centerline{\textbf{Research Objective 3}}}
    \vspace{6pt}
    \large{\centerline{\textbf{Develop data-parallel Monte Carlo methods for estimating probabilities}}}
\end{frame}

\subsection{Back to Working Example: One Initiating Event, Three Fault Trees, Six Basic Events, Five End States}
\begin{frame}
  \begin{columns}
    \column{0.66\textwidth}
      \includesvg[height=\textheight]{1_concepts/dag_pass_3.svg} % Replace with your image file
     \column{0.33\textwidth}
      \includesvg[width=\linewidth]{1_concepts/pra-model.svg} % Replace with your image file
  \end{columns}
\end{frame}


\begin{frame}[t, allowframebreaks]
\frametitle{Monte Carlo Sampling}
\begin{itemize}
  \item Rather than summing or bounding all combinations of failures, \emph{simulate} random draws of \(\mathbf{X}\).
  \item Each Monte Carlo iteration:
    \begin{enumerate}
      \item Sample \(x_1, x_2,\dots,x_n \overset{\text{i.i.d.}}{\sim} \prod p(x_i)\).
      \item Evaluate the Boolean function \(F(\mathbf{x})\) (cost is just logical gate evaluation).
      \item Collect whether \(F(\mathbf{x})=1\) (failure) or 0 (success).
    \end{enumerate}
  \item Repeating for many samples \(\{\mathbf{x}^{(1)}, \dots, \mathbf{x}^{(N)}\}\) yields a \emph{sample average} estimate of the probability.
  \item Benefits:
    \begin{itemize}
      \item Bypasses explicit inclusion-exclusion expansions.
      \item Straightforward to parallelize (evaluate each draw in separate threads or blocks).
    \end{itemize}
\end{itemize}
\end{frame}

\begin{frame}[t, allowframebreaks]
\frametitle{Estimator for the Expected Value (i.e., Probability)}
\begin{itemize}
  \item A Boolean function \(F(\mathbf{x})\) can be viewed as an indicator function: \(F(\mathbf{x}) \in \{0,1\}\).
  \item The event \(\{F(\mathbf{X})=1\}\) has probability \(\mathbb{E}[F(\mathbf{X})]\).
  \item \textbf{Monte Carlo estimator:}
    \[
      \widehat{P}_N
      \;=\;
      \frac{1}{N}\sum_{i=1}^N 
      F\!\bigl(\mathbf{x}^{(i)}\bigr),
    \]
    where each \(\mathbf{x}^{(i)}\) is a random draw from the input distribution.
  \item By the Law of Large Numbers,
    \[
      \lim_{N \to \infty}\;\widehat{P}_N
      \;=\;
      \Pr\bigl[F(\mathbf{X})=1\bigr],
      \quad \text{almost surely}.
    \]
  \item Error decreases at rate \(\mathcal{O}(1/\sqrt{N})\), analyzed via the Central Limit Theorem.
\end{itemize}
\end{frame}