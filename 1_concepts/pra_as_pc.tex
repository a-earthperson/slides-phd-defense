\section{Research Objective 1: Bridge PRA Modeling Semantics with Probabilistic Circuits}
\begin{frame}
    \Large{\centerline{\textbf{Research Objective 1}}}
    \vspace{6pt}
    \large{\centerline{\textbf{Bridge PRA Modeling Semantics with Probabilistic Circuits}}}
\end{frame}

\subsection{PRA Overview}
\begin{frame}[allowframebreaks]
\frametitle{The Triplet Definition of Risk}
\begin{itemize}
  \item Define risk as a set of triplets, each representing:
    \begin{enumerate}
      \item What can go wrong? (\(S_i\))
      \item How likely is it to happen? (\(L_i\))
      \item What are the consequences? (\(X_i\))
    \end{enumerate}
        \vspace{6pt}
  \item
    \begin{equation}
    \label{eq:risk_triplets_slides}
      R \;=\;\bigl\{\langle S_i,\,L_i,\,X_i\rangle\bigr\}_{c},
    \end{equation}
    \(\,c\) represents completeness in enumerating \emph{all} relevant scenarios.
\end{itemize}
\end{frame}

\begin{frame}[allowframebreaks]
\frametitle{Scenario \(S_i\) Modeling in PRA}
\begin{itemize}
  \item Each scenario unfolds from initiating events (IEs), followed by conditional branching events.
          \vspace{6pt}
  \item Fundamental goal: assign probabilities to these scenarios {and} assess resulting outcomes (e.g., core damage, large release).
          \vspace{6pt}
  \item Implementation typically uses structured diagrams such as:
    \begin{itemize}
      \item Event Trees (ETs): forward chaining from IE to various end states.
      \item Fault Trees (FTs): top-down decomposition to basic events (component failures).
    \end{itemize}
\end{itemize}
\end{frame}

\subsection{A Working Example: One Initiating Event, Three Fault Trees, Six Basic Events, Five End States}
\begin{frame}
  \begin{columns}
    % Left column: 2/3 of the text width
    \column{0.5\textwidth}
      \includesvg[height=\textheight]{1_concepts/pra-model.svg} % Replace with your image file

 % Right column: 1/3 of the text width
    \column{0.5\textwidth}
      \begin{table}[t]
        \centering
        \begin{tabular}{ll}
          Variable & Expression \\
          \hline
          % $X_1$ & $A|B'$ \\
          % $X_3$ & $B•C'$ \\
          % $X_2$ & $A'|X_3$ \\
          % $X$   & $X_1•X_2$ \\
          $X$   & $(A|B')•(A'|(B•C'))$ \\
          % $Y_1$ & $D|E$ \\
          $Y$   & $C•(D|E)'$ \\

          % $Z_1$   & $A•C$ \\    
          % $Z_2$   & $D•E$ \\
          $Z$   & $kn[(A•C),(D•E),F']$ \\
          \hline
        \end{tabular}
        \caption{Unsimplified Boolean expression for each Top Event}
      \end{table}
      Small, but non-trivial structure:
      \begin{itemize}
          \item {Basic events are shared.}
          \item {Some gate outputs are negated.}
          \item {Event $Z$ is a (k=2) of n=3 gate.}
      \end{itemize}
  \end{columns}
\end{frame}

\begin{frame}{Compile a Directed Acylic Graph (DAG) from Logic Model}
  \begin{columns}
    % Left column: 2/3 of the text width
    \column{0.33\textwidth}
      \includesvg[width=\linewidth]{1_concepts/pra-model.svg} % Replace with your image file

 % Right column: 1/3 of the text width
     \column{0.33\textwidth}
      \includesvg[width=\linewidth]{1_concepts/dag_pass_1.svg}\par % Replace with your image file

     \column{0.34\textwidth}
      Start with an Arbitrary Topological Ordering:
      \begin{itemize}
          \item {Aim for succinctness.}
          \item {prefer HW-native ops.}
          \item {e.g. don't expand k/n.}
      \end{itemize}
  \end{columns}
\end{frame}

\begin{frame}
  \begin{columns}
    \column{0.60\textwidth}
    {
        \includesvg[height=0.5\textheight]{1_concepts/dag_pass_1.svg}\par
        \vspace{10pt}
        \textbf{Refinements:}
      \begin{itemize}
          \item {Partition into layers.}
          \item {Vectorize for SIMD by fusing similar ops.}
          \item {Feed-forward only: improves caching.}
      \end{itemize}\par
      \vspace{2pt}
      \tiny{\textbf{\emph{Still not probabilistic: inputs, outputs are bitpacked INT64 tensors.}}}
    }
     \column{0.4\textwidth}
      \includesvg[height=1.05\textheight]{1_concepts/dag_pass_2.svg}\par % Replace with your image file
  \end{columns}
\end{frame}

\subsection{Knowledge Compilation and Queries}
\begin{frame}[t]{Querying the Compiled Knowledge Graph}
  \begin{columns}
    \column{0.60\textwidth}
    {
      \textbf{The Simplest Type of Query: Eval(G)}\par
      \begin{itemize}
          \item {Set the inputs [on/off].}
          \item {Observe the outputs [on/off].}
          \item {Can be used as a building block for an embedding ML model.}
      \end{itemize}\par
      \vspace{10pt}
      \textbf{But just how fast is it?}
    }
     \column{0.4\textwidth}
      \includesvg[height=0.9\textheight]{1_concepts/dag_pass_2.svg}\par % Replace with your image file
  \end{columns}
\end{frame}


\begin{frame}[t]{Eval Query on the Compiled Knowledge Graph}
  \begin{columns}
    \column{0.60\textwidth}
    {
      \textbf{Eval Query Performance on GPUs:}
      \begin{itemize}
          \item {Latency: 200-300 $ms$ per graph pass.}
          \item {Throughput: VRAM bound (see plot).}
          \item {Benchmarked on Nvidia GTX 1660 [6GB].}
          \item {Graph sizes: from $\approx 50$ to $\approx 2000$ nodes.}
          \item {Evals: from 16M to 1B per node per pass.}
      \end{itemize}
      \vspace{10pt}
      \textbf{Q: Are these enough samples to estimate the Expectation Query?}
    }
     \column{0.4\textwidth}
        \centering
        \includesvg[height=0.9\textheight]{1_concepts/mem_allocation_lines_zoom.svg}
  \end{columns}
\end{frame}

\begin{frame}{Estimator for the Expected Value (i.e., Probability)}
\begin{itemize}
  \item A Boolean function \(F(\mathbf{x})\) can be viewed as an indicator function: \(F(\mathbf{x}) \in \{0,1\}\).
  \item The event \(\{F(\mathbf{X})=1\}\) has probability \(\mathbb{E}[F(\mathbf{X})]\).
  \item \textbf{Monte Carlo estimator:}
    \[
      \widehat{P}_N
      \;=\;
      \frac{1}{N}\sum_{i=1}^N 
      F\!\bigl(\mathbf{x}^{(i)}\bigr),
    \]
    where each \(\mathbf{x}^{(i)}\) is a random draw from the input distribution.
  \item By the Law of Large Numbers,
    \[
      \lim_{N \to \infty}\;\widehat{P}_N
      \;=\;
      \Pr\bigl[F(\mathbf{X})=1\bigr],
      \quad \text{almost surely}.
    \]
  \item Error decreases at rate \(\mathcal{O}(1/\sqrt{N})\), analyzed via the Central Limit Theorem.
\end{itemize}
\end{frame}

\begin{frame}{Monte Carlo Sampling}
\begin{itemize}
  \item Rather than summing or bounding all combinations of failures, \emph{simulate} random draws of \(\mathbf{X}\).
  \item Each Monte Carlo iteration:
    \begin{enumerate}
      \item Sample \(x_1, x_2,\dots,x_n \overset{\text{i.i.d.}}{\sim} \prod p(x_i)\).
      \item Evaluate the Boolean function \(F(\mathbf{x})\) (cost is just logical gate evaluation).
      \item Collect whether \(F(\mathbf{x})=1\) (failure) or 0 (success).
    \end{enumerate}
  \item Repeating for many samples \(\{\mathbf{x}^{(1)}, \dots, \mathbf{x}^{(N)}\}\) yields a \emph{sample average} estimate of the probability.
  \item Benefits:
    \begin{itemize}
      \item Bypasses explicit inclusion-exclusion expansions.
      \item Straightforward to parallelize (evaluate each draw in separate threads or blocks).
    \end{itemize}
\end{itemize}
\end{frame}


\subsection{Back to Working Example: One Initiating Event, Three Fault Trees, Six Basic Events, Five End States}
\begin{frame}
  \begin{columns}
    \column{0.66\textwidth}
      \includesvg[height=\textheight]{1_concepts/dag_pass_3.svg} % Replace with your image file
     \column{0.33\textwidth}
      \includesvg[width=\linewidth]{1_concepts/pra-model.svg} % Replace with your image file
  \end{columns}
\end{frame}
