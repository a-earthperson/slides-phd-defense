

\begin{frame}[allowframebreaks]{Research Contribution}
\begin{enumerate}
\item \textbf{Unification within the generalized Probabilistic Circuit (PC) framework:}
  \begin{itemize}
    \item{Unify linked event/fault tree logic as a special class of PCs.}
    \item {Provide tractability and accuracy guarantees.}
  \end{itemize}
\vspace{16pt}
\item \textbf{Data-Parallel Monte Carlo for Boolean Circuits:}
  \begin{itemize}
    \item{Simultaneous evaluation of \emph{all} intermediate gates, success, and failure paths.}
    \item{Relax coherence constraints - arbitrary shapes with NOT gates permitted.}
    \item{Vectorized bitwise hardware ops for logical primitives (AND, OR, XOR, etc.)}
    \item{Specialized treatment of \(k/n\) logic, without expansion.}
    \item{Simultaneous use of all available compute - GPUs, multicore CPUs.}
    \vspace{16pt}
  \end{itemize}
\framebreak

\item \textbf{Benchmarks and Open-source Prototypes:}  
  \begin{itemize}
    \item{Preliminary benchmarks on Aralia dataset.}
    \item{Ongoing benchmarks on generic PWR PRA model.}
    \item{C++/SYCL implementation, integrated into widely available SCRAM PRA tool.}
    \item{Alternative Python implementation, built using Tensorflow, available on PyPI.}
\vspace{16pt}
  \end{itemize}


\item \textbf{Sampling Techniques for Partial-Derivatives:}  
  \begin{itemize}
    \item{Bitwise algorithm for computing derivatives using Monte Carlo sampling.}
    \item{Sensitivity and importance analysis using sampling methods.}
    \item{Gradient computation opens a path towards learning-based tasks.}
    \vspace{4pt}
  \end{itemize}
\end{enumerate}
\end{frame}
