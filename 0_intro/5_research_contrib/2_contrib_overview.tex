% \begin{frame}[allowframebreaks]{Research Contribution}
% \begin{enumerate}
% \item \textbf{Bridge PRA modeling semantics with Probabilistic Circuits}
% \vspace{4pt}
% \item \textbf{Develop data-parallel Monte Carlo methods for evaluating Boolean circuits}
% \vspace{4pt}
% \item \textbf{Open-source implementations and benchmarks}
% \vspace{4pt}
% \item \textbf{Develop Monte Carlo sampling techniques for computing partial-derivatives on Boolean circuits}  
% \end{enumerate}
% \end{frame}

%\item \textbf{Data-Parallel Monte Carlo for Expectation Queries over Boolean Circuits}

\begin{frame}{This Dissertation Contributes}
\begin{enumerate}[<+->]
  \item \textbf{Unified Risk Graph.}  Formalized PRA models as probabilistic circuits, prove semantic equivalence.
  \item \textbf{Hardware–Native Gate Set.}  Population-count kernels for $k$-of-$n$ and cardinality gates :: exponential graph compression.
  \item \textbf{MC-Oriented Knowledge Compilation.}  Optimizes PDAGs for throughput.
  \item \textbf{Bit-Parallel Monte-Carlo Engine.}  SYCL kernels achieving massive parallelism.
  \item \textbf{Rigorous Convergence Criteria.}  Multiobjective, with formal error bounds.
  \item \textbf{Domain Extensions.}  Common-cause failures, importance measures, and an importance-sampling prototype for rare events.
  \item \textbf{Open-Source Release \& Benchmarks.}  Reproducible evaluation on 43 Aralia models; code under permissive license.
\end{enumerate}
\end{frame}